% Options for packages loaded elsewhere
\PassOptionsToPackage{unicode}{hyperref}
\PassOptionsToPackage{hyphens}{url}
%
\documentclass[
]{article}
\title{Rats : A normal hierarchical model}
\author{Vincent Guitteny, Freddie Joly, Tom Léchappé et Elyes Zribi}
\date{01 avril 2022}

\usepackage{amsmath,amssymb}
\usepackage{lmodern}
\usepackage{iftex}
\ifPDFTeX
  \usepackage[T1]{fontenc}
  \usepackage[utf8]{inputenc}
  \usepackage{textcomp} % provide euro and other symbols
\else % if luatex or xetex
  \usepackage{unicode-math}
  \defaultfontfeatures{Scale=MatchLowercase}
  \defaultfontfeatures[\rmfamily]{Ligatures=TeX,Scale=1}
\fi
% Use upquote if available, for straight quotes in verbatim environments
\IfFileExists{upquote.sty}{\usepackage{upquote}}{}
\IfFileExists{microtype.sty}{% use microtype if available
  \usepackage[]{microtype}
  \UseMicrotypeSet[protrusion]{basicmath} % disable protrusion for tt fonts
}{}
\makeatletter
\@ifundefined{KOMAClassName}{% if non-KOMA class
  \IfFileExists{parskip.sty}{%
    \usepackage{parskip}
  }{% else
    \setlength{\parindent}{0pt}
    \setlength{\parskip}{6pt plus 2pt minus 1pt}}
}{% if KOMA class
  \KOMAoptions{parskip=half}}
\makeatother
\usepackage{xcolor}
\IfFileExists{xurl.sty}{\usepackage{xurl}}{} % add URL line breaks if available
\IfFileExists{bookmark.sty}{\usepackage{bookmark}}{\usepackage{hyperref}}
\hypersetup{
  pdftitle={Rats : A normal hierarchical model},
  pdfauthor={Vincent Guitteny, Freddie Joly, Tom Léchappé et Elyes Zribi},
  hidelinks,
  pdfcreator={LaTeX via pandoc}}
\urlstyle{same} % disable monospaced font for URLs
\usepackage[margin=1in]{geometry}
\usepackage{longtable,booktabs,array}
\usepackage{calc} % for calculating minipage widths
% Correct order of tables after \paragraph or \subparagraph
\usepackage{etoolbox}
\makeatletter
\patchcmd\longtable{\par}{\if@noskipsec\mbox{}\fi\par}{}{}
\makeatother
% Allow footnotes in longtable head/foot
\IfFileExists{footnotehyper.sty}{\usepackage{footnotehyper}}{\usepackage{footnote}}
\makesavenoteenv{longtable}
\usepackage{graphicx}
\makeatletter
\def\maxwidth{\ifdim\Gin@nat@width>\linewidth\linewidth\else\Gin@nat@width\fi}
\def\maxheight{\ifdim\Gin@nat@height>\textheight\textheight\else\Gin@nat@height\fi}
\makeatother
% Scale images if necessary, so that they will not overflow the page
% margins by default, and it is still possible to overwrite the defaults
% using explicit options in \includegraphics[width, height, ...]{}
\setkeys{Gin}{width=\maxwidth,height=\maxheight,keepaspectratio}
% Set default figure placement to htbp
\makeatletter
\def\fps@figure{htbp}
\makeatother
\setlength{\emergencystretch}{3em} % prevent overfull lines
\providecommand{\tightlist}{%
  \setlength{\itemsep}{0pt}\setlength{\parskip}{0pt}}
\setcounter{secnumdepth}{5}
\usepackage{stmaryrd} \usepackage{float} \floatplacement{figure}{H}
\ifLuaTeX
  \usepackage{selnolig}  % disable illegal ligatures
\fi

\begin{document}
\maketitle

\newenvironment{cols}[1][]{}{}
\newenvironment{col}[1]{\begin{minipage}{#1}\ignorespaces}{%
\end{minipage}
\ifhmode\unskip\fi
\aftergroup\useignorespacesandallpars}
\def\useignorespacesandallpars#1\ignorespaces\fi{%
#1\fi\ignorespacesandallpars}
\makeatletter
\def\ignorespacesandallpars{%
  \@ifnextchar\par
    {\expandafter\ignorespacesandallpars\@gobble}%
    {}%
}
\makeatother

\renewcommand\contentsname{Table des matières}
\newpage
\tableofcontents
\newpage

\hypertarget{pruxe9sentation-du-jeu-de-donnuxe9es}{%
\section{Présentation du jeu de
données}\label{pruxe9sentation-du-jeu-de-donnuxe9es}}

Nous disposons des poids de 30 jeunes rats, mesurés chaque semaine
pendant 5 semaines. La dimension de notre jeu de données est donc de 30
par 5. Nos variables \(x_j,j=1,…,5\) correspondent aux différents âges
des rats, en jour (\(x_j = {8,15,22,29,36}\)), et nos données \(Y_{ij}\)
correspondent au poids du rat \(i\) à l'âge \(x_j\).

Un tracé des 30 courbes de croissance suggère des signes de courbure
vers le bas :

\begin{center}\includegraphics{Rats---A-normal-hierarchical-model_files/figure-latex/unnamed-chunk-1-1} \end{center}

\hypertarget{pruxe9sentation-du-moduxe8le}{%
\section{Présentation du modèle}\label{pruxe9sentation-du-moduxe8le}}

Le modèle est essentiellement une courbe de croissance linéaire à effets
aléatoires :

\(Y_{ij} \sim \mathcal N(\alpha_i + \beta_i(x_j - \bar x), \sigma_c ^2)\)
où \(\bar x = 22\) et \(\sigma_c ^2 \sim InvGamma(a_c,b_c)\)
(\(\sigma_c ^2\) a une loi a priori non informative)

\(\alpha_i \sim \mathcal N(\alpha_c,\sigma_{\alpha}^2)\)

\(\beta_i \sim \mathcal N(\beta_c,\sigma_{\beta}^2)\).

On note l'absence de paramètre représentant la corrélation entre
\(\alpha_i\) et \(\beta_i\). Pour l'instant, nous standardisons les
\(x_j\) autour de leur moyenne pour réduire la dépendance entre
\(\alpha_i\) et \(\beta_i\) dans leur vraisemblance : en fait, pour les
données entièrement équilibrées (centrées et réduites), une indépendance
complète est atteinte (notons qu'en général, l'indépendance a priori
n'oblige pas les distributions a posteriori à être indépendantes).

Les paramètres
\(\alpha_c, \sigma_{\alpha}^2, \beta_c, \sigma_{\beta}^2\) ont des loi a
priori « non informatives » et indépendantes :

\(\alpha_c \sim\mathcal N(0,\sigma_a^2)\)

\(\sigma_{\alpha}^2 \sim InvGamma(a_{\alpha},b_{\alpha})\)

\(\beta_c \sim \mathcal N(0,\sigma_b^2)\)

\(\sigma_{\beta}^2 \sim InvGamma(a_{\beta},b_{\beta})\).

L'utilisation d'un tel modèle hiérarchique dans ce cas peut s'expliquer
par le fait que le poids d'un rat dépend de plusieurs facteurs. En effet
on peut imaginer que le poids du rat varie en fonction du rat lui-même
(par exemple en fonction de son sexe), en fonction de son alimentation
ou encore en fonction de son environnement. Plus de détails sur
l'expérience aurait pu nous indiquer les informations exactes pour ce
modèle hiérarchique. Ci-dessous le modèle obtenu sous forme de graphes.

\begin{figure}
\centering
\includegraphics[width=0.5\textwidth,height=\textheight]{Model.jpg}
\caption{Modèle hierarchique}
\end{figure}

\hypertarget{calculs-des-lois-conditionnelles-pleines}{%
\section{Calculs des lois conditionnelles
pleines}\label{calculs-des-lois-conditionnelles-pleines}}

Afin de mettre en oeuvre un algorithme MCMC (Monte Carlo Markov Chain),
à l'aide de la fonction \texttt{mcmc} du package \texttt{coda}, sur une
chaîne de Markov obtenue par un échantilloneur de Gibbs, nous avons eu
la nécessité de calculer les lois conditionnelles pleines (loi
conditionnellement aux autres paramètres) des paramètres \(\alpha_c\),
\(\beta_c\), \(\sigma_{\alpha}^2\), \(\sigma_{\beta}^2\),
\(\sigma_c ^2\), \(\alpha_i\) et \(\beta_i\) (et de voir si celles-ci
étaient identifiables). Les paramètres \(\alpha_c\), \(\beta_c\),
\(\alpha_i\) et \(\beta_i\) ont des lois a priori gaussiennes et les
paramètres \(\sigma_{\alpha}^2\), \(\sigma_{\beta}^2\) et
\(\sigma_c ^2\) ont des lois a priori inverses gamma. Nous donnons
ci-dessous un exemple de calcul pour une loi gaussienne et un exemple
pour une loi inverse gamma (le reste des démonstrations se trouve en
annexe).

\hypertarget{exemple-de-calcul-avec-des-lois-gaussiennes}{%
\subsection{Exemple de calcul avec des lois
gaussiennes}\label{exemple-de-calcul-avec-des-lois-gaussiennes}}

Dans cet exemple, on cherche à calculer la loi conditionnelle pleine du
paramètre \(\alpha_c\). Pour cela, nous avons besoin de la loi a priori
de \(\alpha_{c} \sim \mathcal{N}(O,\,\sigma_{a}^{2})\) et de la loi de
\(\alpha_{i} \sim \mathcal{N}(\alpha_{c},\,\sigma_{\alpha}^{2})\).

\begin{align*}
\Pi(\alpha_{c}|...) &\propto \Pi(\alpha_{c}) \prod_{i=1}^{n_{i}} \Pi(\alpha_{i}|\alpha_{c},\sigma_{\alpha}^{2} ) \\
        &\propto \exp\left(\frac{-\alpha_{c}^{2}}{2\sigma_{a}^{2}}\right)\prod_{i=1}^{n_{i}} \exp\left(\frac{-(\alpha_{i}-\alpha_{c})^{2}}{2\sigma_{\alpha}^{2}}\right) \\
        &\propto \exp\left(\frac{-\alpha_{c}^{2}}{2\sigma_{a}^{2}}\right)\prod_{i=1}^{n_{i}} \exp\left(\frac{-(\alpha_{c}^{2}-2\alpha_{i}\alpha_{c})}{2\sigma_{\alpha}^{2}}\right) \\
        &\propto \exp\left(\frac{-\alpha_{c}^{2}}{2\sigma_{a}^{2}}\right) \exp\left(\frac{-n_i\alpha_{c}^{2}+2\alpha_{c}\sum\limits_{i=1}^{n_{i}} \alpha_{i}}{2\sigma_{\alpha}^{2}}\right)\\
        &\propto \exp\left(\frac{-\alpha_{c}^{2}(\sigma_{\alpha}^{2}+n_i\sigma_{a}^{2}) +2\alpha_{c}\sigma_{a}^{2}\sum\limits_{i=1}^{n_{i}} \alpha_{i}}{2\sigma_{a}^{2}\sigma_{\alpha}^{2}}\right)\\
\end{align*}

Ainsi, on identifie une loi gaussienne :

\begin{align*}
\Pi(\alpha_{c}|...)    &\sim \mathcal{N}\left(\frac{\sigma_{a}^{2}\sum\limits_{i=1}^{n_{i}} \alpha_{i}}{\sigma_{\alpha}^{2}+n_{i}\sigma_{a}^{2}},\frac{\sigma_{a}^{2}\sigma_{\alpha}^{2}}{\sigma_{\alpha}^{2}+n_{i}\sigma_{a}^{2}}\right)
\end{align*}

\hypertarget{exemple-de-calcul-avec-des-lois-inverses-gamma}{%
\subsection{Exemple de calcul avec des lois inverses
gamma}\label{exemple-de-calcul-avec-des-lois-inverses-gamma}}

Ici nous calculons la loi conditionnelle pleine du paramètre
\(\sigma_{\alpha}^2\). Pour cela, nous avons besoin de la loi a priori
de \(\sigma_{\alpha}^{2} \sim InvGamma(a_{\alpha},b_{\alpha})\) et de la
loi de
\(\alpha_{i} \sim \mathcal{N}(\alpha_{c},\,\sigma_{\alpha}^{2})\).

\begin{align*}
\Pi(\sigma_{\alpha}^{2}|...) &\propto \Pi(\sigma_{\alpha}^{2}) \prod_{i=1}^{n_{i}}\Pi(\alpha_{i}|\alpha_{c},\sigma_{\alpha}^{2}) \\
&\propto \exp^{\frac{-b_{\alpha}}{\sigma_{\alpha}^{2}}}(\sigma_{\alpha}^{2})^{-a_{\alpha}-1}\prod_{i=1}^{n_{i}} \exp\left(\frac{-(\alpha_{i}-\alpha_{c})^{2}}{2\sigma_{\alpha}^{2}}\right)(\sigma_{\alpha}^{2})^{-\frac{1}{2}}\\
&\propto (\sigma_{\alpha}^{2})^{-a_{\alpha}-1-\frac{n_{i}}{2}}\exp\left(\frac{-2b_{\alpha}-\sum\limits_{i=1}^{n_{i}}(\alpha_{i}-\alpha_{c})^{2}}{2\sigma_{\alpha}^{2}}\right)\\
\end{align*}

Ainsi, on identifie une loi inverse gamma :

\begin{align*}
\Pi(\sigma_{\alpha}^{2}|...) &\sim InvGamma(\frac{n_{i}}{2}+a_{\alpha},\frac{2b_{\alpha}+\sum\limits_{i=1}^{n_{i}}(\alpha_{i}-\alpha_{c})^{2}}{2})
\end{align*}

\hypertarget{impluxe9mentation-et-ruxe9sultats}{%
\section{Implémentation et
résultats}\label{impluxe9mentation-et-ruxe9sultats}}

Maintenant que nous avons obtenues et identifiées toutes nos lois
conditionnelles pleines de nos paramètres, nous avons pu implémenter un
échantilloneur de Gibbs et utiliser un algorithme MCMC sur la chaîne de
Markov obtenue par cet échantilloneur.

Afin de comparer nos résultats avec ceux de l'expérience, nous avons
utilisé un ``burnin'' de 1000 sur la chaîne de Markov de taille totale
10 000 qui est retourné par notre fonction simulant un échantilloneur de
Gibbs.

On observe que l'on retrouve la même moyenne que ceux obtenus par les
auteurs pour les paramètres \(\alpha_0\) (défini par
\(\alpha_0 = \alpha_c - \bar{x}\beta_c\)) et \(\beta_c\) dont on donne
les valeurs dans le tableau ci-dessous.

\begin{longtable}[]{@{}cc@{}}
\toprule
& Moyenne \\
\midrule
\endhead
\(\alpha_0\) & 106.568 \\
\(\beta_c\) & 6.186 \\
\(\sigma_c^2\) & 0.003 \\
\bottomrule
\end{longtable}

\begin{center}
Table 1 : Moyenne des paramètres $\alpha_0$, $\beta_c$ et $\sigma_c^2$.
\end{center}

Enfin, nous avons voulu comparer nos résultats en faisant une prédiction
du poids du rat 26 et en comparant nos résultats avec ceux du jeu de
données initial.

\begin{longtable}[]{@{}ccc@{}}
\toprule
& Valeurs prédites & Valeurs réelles \\
\midrule
\endhead
Semaine 2 & 207 & 207 \\
Semaine 3 & 254 & 257 \\
Semaine 4 & 300 & 303 \\
Semaine 5 & 347 & 345 \\
\bottomrule
\end{longtable}

\begin{center}
Table 2 : Valeurs prédites et réelles du poids du rat 26.
\end{center}

On retrouve bien dans ce tableau que nos valeurs prédites sont très
proches des valeurs réelles pour les différents poids de ce rat.

L'ensemble de notre code est disponible dans le fichier Gibbs.R sur la
page Github :
\url{https://github.com/VincentGuitteny/Rats-a-normal-hierarchical-model}

\newpage

\hypertarget{annexes}{%
\section{Annexes}\label{annexes}}

\hypertarget{loi-conditionnelle-pleine-du-paramuxe8tre-beta_c}{%
\subsection{\texorpdfstring{Loi conditionnelle pleine du paramètre
\(\beta_c\)}{Loi conditionnelle pleine du paramètre \textbackslash beta\_c}}\label{loi-conditionnelle-pleine-du-paramuxe8tre-beta_c}}

\begin{align*}
\Pi(\beta_{c}|...) &\propto \Pi(\beta_{c}) \prod_{i=1}^{n_{i}} \Pi(\beta_{i}|\beta_{c},\sigma_{\beta}^{2} ) \\
        &\propto \exp\left(\frac{-\beta_{c}^{2}}{2\sigma_{b}^{2}}\right)\prod_{i=1}^{n_{i}} \exp\left(\frac{-(\beta_{i}-\beta_{c})^{2}}{2\sigma_{\beta}^{2}}\right) \\
        &\propto \exp\left(\frac{-\beta_{c}^{2}}{2\sigma_{b}^{2}}\right)\prod_{i=1}^{n_{i}} \exp\left(\frac{-(\beta_{c}^{2}-2\beta_{i}\beta_{c})}{2\sigma_{\beta}^{2}}\right) \\
        &\propto \exp\left(\frac{-\beta_{c}^{2}}{2\sigma_{b}^{2}}\right) \exp\left(\frac{-n_i\beta_{c}^{2}+2\beta_{c}\sum\limits_{i=1}^{n_{i}} \beta_{i}}{2\sigma_{\beta}^{2}}\right)\\
        &\propto \exp\left(\frac{-\beta_{c}^{2}(\sigma_{\beta}^{2}+n_i\sigma_{b}^{2}) +2\beta_{c}\sigma_{b}^{2}\sum\limits_{i=1}^{n_{i}} \beta_{i}}{2\sigma_{b}^{2}\sigma_{\beta}^{2}}\right)\\
        &\sim \mathcal{N}\left(\frac{\sigma_{b}^{2}\sum\limits_{i=1}^{n_{i}} \beta_{i}}{\sigma_{\beta}^{2}+n_{i}\sigma_{b}^{2}},\frac{\sigma_{b}^{2}\sigma_{\beta}^{2}}{\sigma_{\beta}^{2}+n_{i}\sigma_{b}^{2}}\right)
\end{align*}

\hypertarget{loi-conditionnelle-pleine-du-paramuxe8tre-alpha_i}{%
\subsection{\texorpdfstring{Loi conditionnelle pleine du paramètre
\(\alpha_i\)}{Loi conditionnelle pleine du paramètre \textbackslash alpha\_i}}\label{loi-conditionnelle-pleine-du-paramuxe8tre-alpha_i}}

\begin{align*}
\Pi(\alpha_{i}|...) &\propto \Pi(\alpha_{i}|\alpha_{c},\sigma_{\alpha}^{2} ) \prod_{j=1}^{n_{j}} \Pi(Y_{ij}|\alpha_{i},\beta_{i},\sigma_{c}^{2} ) \\
        &\propto \exp\left(\frac{-(\alpha_{i}-\alpha_{c})^{2}}{2\sigma_{\alpha}^{2}}\right) \prod_{j=1}^{n_{j}} \exp\left(\frac{-(y_{ij}-\alpha_{i}-\beta_i(x_{j}-\bar{x}))^{2}}{2\sigma_{c}^{2}}\right) 
\end{align*}

En posant \(\mu_{ij} = y_{ij} - \beta_i(x_{j}-\bar{x})\), on a

\begin{align*}
\Pi(\alpha_{i}|...)&\propto \exp\left(\frac{-\alpha_i^2+2\alpha_i\alpha_c}{2\sigma_{\alpha}^{2}}\right) \prod_{j=1}^{n_{j}} \exp\left(\frac{-(\alpha_{i}-\mu_{ij})^{2}}{2\sigma_{c}^{2}}\right)  \\
        &\propto \exp\left(\frac{-\alpha_i^2+2\alpha_i\alpha_c}{2\sigma_{\alpha}^{2}}\right) \prod_{j=1}^{n_{j}} \exp\left(\frac{-\alpha_i^2+2\alpha_i\mu_{ij}}{2\sigma_{c}^{2}}\right)  \\
        &\propto \exp\left(\frac{-\alpha_i^2(\sigma_c^2+n_j\sigma_{\alpha}^2) +2\alpha_i(\alpha_c\sigma_c^2+\sigma_{\alpha}^2\sum\limits_{j=1}^{n_j} \mu_{ij})}{2\sigma_{\alpha}^{2}\sigma_c^2}\right)\\
        &\sim \mathcal{N}\left(\frac{\alpha_c\sigma_c^2+\sigma_{\alpha}^2\sum\limits_{j=1}^{n_j} \mu_{ij}}{\sigma_c^2+n_j\sigma_{\alpha}^2},\frac{\sigma_{\alpha}^{2}\sigma_c^2}{\sigma_c^2+n_j\sigma_{\alpha}^2}\right)
\end{align*}

\hypertarget{loi-conditionnelle-pleine-du-paramuxe8tre-beta_i}{%
\subsection{\texorpdfstring{Loi conditionnelle pleine du paramètre
\(\beta_i\)}{Loi conditionnelle pleine du paramètre \textbackslash beta\_i}}\label{loi-conditionnelle-pleine-du-paramuxe8tre-beta_i}}

\begin{align*}
\Pi(\beta_{i}|...) &\propto \Pi(\beta_{i}|\beta_{c},\sigma_{\beta}^{2} ) \prod_{j=1}^{n_{j}} \Pi(Y_{ij}|\alpha_{i},\beta_{i},\sigma_{c}^{2} ) \\
        &\propto \exp\left(\frac{-(\beta_{i}-\beta_{c})^{2}}{2\sigma_{\beta}^{2}}\right) \prod_{j=1}^{n_{j}} \exp\left(\frac{-(y_{ij}-\alpha_{i}-\beta_i(x_{j}-\bar{x}))^{2}}{2\sigma_{c}^{2}}\right) 
\end{align*}

En posant \(\mu_{ij} = y_{ij} - \alpha_i\), on a

\begin{align*}
\Pi(\beta_{i}|...)&\propto \exp\left(\frac{-\beta_i^2+2\beta_i\beta_c}{2\sigma_{\beta}^{2}}\right) \prod_{j=1}^{n_{j}} \exp\left(\frac{-(\beta_{i}(x_{j}-\bar{x})-\mu_{ij})^{2}}{2\sigma_{c}^{2}}\right)  \\
        &\propto \exp\left(\frac{-\beta_i^2+2\beta_i\beta_c}{2\sigma_{\beta}^{2}}\right) \prod_{j=1}^{n_{j}} \exp\left(\frac{-\beta_i^2(x_{j}-\bar{x})^2+2\beta_i(x_{j}-\bar{x})\mu_{ij}}{2\sigma_{c}^{2}}\right)  \\
        &\propto \exp\left(\frac{-\beta_i^2(\sigma_c^2+\sigma_{\beta}^2\sum\limits_{j=1}^{n_j}(x_{j}-\bar{x})^2) +2\beta_i(\beta_c\sigma_c^2+\sigma_{\beta}^2\sum\limits_{j=1}^{n_j} (x_{j}-\bar{x})\mu_{ij})}{2\sigma_{\beta}^{2}\sigma_c^2}\right)\\
        &\sim \mathcal{N}\left(\frac{\beta_c\sigma_c^2+\sigma_{\beta}^2\sum\limits_{j=1}^{n_j} (x_{j}-\bar{x})\mu_{ij}}{\sigma_c^2+\sigma_{\beta}^2\sum\limits_{j=1}^{n_j}(x_{ij}-\bar{x})^2},\frac{\sigma_{\beta}^{2}\sigma_c^2}{\sigma_c^2+\sigma_{\beta}^2\sum\limits_{j=1}^{n_j}(x_{j}-\bar{x})^2}\right)
\end{align*}

\hypertarget{loi-conditionnelle-pleine-du-paramuxe8tre-sigma_beta2}{%
\subsection{\texorpdfstring{Loi conditionnelle pleine du paramètre
\(\sigma_{\beta}^2\)}{Loi conditionnelle pleine du paramètre \textbackslash sigma\_\{\textbackslash beta\}\^{}2}}\label{loi-conditionnelle-pleine-du-paramuxe8tre-sigma_beta2}}

\begin{align*}
\Pi(\sigma_{\beta}^{2}|...) &\propto \Pi(\sigma_{\beta}^{2}) \prod_{i=1}^{n_{i}}\Pi(\beta_{i}|\beta_{c},\sigma_{\beta}^{2}) \\
&\propto \exp^{\frac{-b_{\beta}}{\sigma_{\beta}^{2}}}(\sigma_{\beta}^{2})^{-a_{\beta}-1}\prod_{i=1}^{n_{i}} \exp\left(\frac{-(\beta_{i}-\beta_{c})^{2}}{2\sigma_{\beta}^{2}}\right)(\sigma_{\beta}^{2})^{-\frac{1}{2}}\\
&\propto (\sigma_{\beta}^{2})^{-a_{\beta}-1-\frac{n_{i}}{2}}\exp\left(\frac{-2b_{\beta}-\sum\limits_{i=1}^{n_{i}}(\beta_{i}-\beta_{c})^{2}}{2\sigma_{\beta}^{2}}\right)\\
&\sim InvGamma(\frac{n_{i}}{2}+a_{\beta},\frac{2b_{\beta}+\sum\limits_{i=1}^{n_{i}}(\beta_{i}-\beta_{c})^{2}}{2})
\end{align*}

\hypertarget{loi-conditionnelle-pleine-du-paramuxe8tre-sigma_c2}{%
\subsection{\texorpdfstring{Loi conditionnelle pleine du paramètre
\(\sigma_{c}^2\)}{Loi conditionnelle pleine du paramètre \textbackslash sigma\_\{c\}\^{}2}}\label{loi-conditionnelle-pleine-du-paramuxe8tre-sigma_c2}}

\begin{align*}
\Pi(\sigma_c^2|...) &\propto \Pi(\sigma_c^2) \prod_{i=1}^{n_i}\prod_{j=1}^{n_j}\Pi(Y_{ij}|\alpha_{i},\beta_{i} \sigma_c^2)
\end{align*}

En posant \(\mu_{ij} = \alpha_i + \beta_i(x_j-\bar{x})\), on a

\begin{align*}
\Pi(\sigma_c^2|...) &\propto \exp^{\frac{-b_c}{\sigma_c^2}}(\sigma_c^2)^{-a_c-1}\prod_{i=1}^{n_i}\prod_{j=1}^{n_j} \exp\left(\frac{-(y_{ij}-\mu_{ij})^{2}}{2\sigma_c^2}\right)(\sigma_c^2)^{-\frac{1}{2}}\\
&\propto (\sigma_c^2)^{-a_c-1-\frac{n}{2}}\exp\left(\frac{-2b_c-\sum\limits_{i=1}^{n_i}\sum\limits_{j=1}^{n_j}(y_{ij}-\mu_{ij})^{2}}{2\sigma_c^2}\right)\\
&\sim InvGamma(\frac{n}{2}+a_c,\frac{2b_c+\sum\limits_{i=1}^{n_i}\sum\limits_{j=1}^{n_j}(y_{ij}-\mu_{ij})^{2}}{2})
\end{align*}

\end{document}
